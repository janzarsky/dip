\chapter{Storage Medium}
\label{storage}

The contents of the enclosed CD:
\begin{description}
    \item [\texttt{RoadRunner/}] The source code of the analyzer.
    \item [\texttt{RoadRunner-compiled/}] Compiled analyzer.
    \item [\texttt{patches/}] The source code of the analyzer, as patches
        against commit \texttt{b141616} in the upstream RoadRunner
        repository\footnote{available at
        \url{https://github.com/stephenfreund/RoadRunner}}.
    \item [\texttt{contracts-for-concurrency.pdf}] The text of the
        thesis.
    \item [\texttt{contracts-for-concurrency-print.pdf}] The text of
        the thesis, for color printing.
    \item [\texttt{contracts-for-concurrency/}] The sources code of the text of
        the thesis.
\end{description}

\chapter{Manual}
\label{manual}

The analyzer requires a~Java Virtual Machine version 11 and
Ant\footnote{available at \url{http://ant.apache.org/}}. For
convenience, the analyzer should be compiled and run on Linux as there are
several Bash scripts used in the process. All commands, such as \texttt{ant} or
\texttt{javac}, that are run during the compilation should belong to the same
JVM installation.

On Fedora 34, Java 11 is used by default. Ant can be installed by running
\texttt{dnf install ant} as root. On Ubuntu 21.04, Java 11 must be installed by
running \texttt{apt install openjdk-11-jdk}. Ant must be installed locally from
the project's website.

To compile the project, run \texttt{ant} in the RoadRunner directory. The
project should compile and print \texttt{BUILD SUCCESSFUL} at the end. The unit
tests can be run at this step with \texttt{ant test}. The tests should all pass.

Before running RoadRunner, edit the \texttt{msetup} file. On line 36, edit the
path to the JVM installation. On Fedora 34, \texttt{/usr/lib/jvm/java-11} should
be used. On Ubuntu 21.04, \texttt{/usr/lib/jvm/java-11-openjdk-amd64} should be
used. Then run \texttt{source msetup}. The environment variables should be
properly exported.

To verify the compilation, run \texttt{rrrun -help}. To instrument and run a
testing program, first compile the program: \texttt{javac test/Test.java} and
then run RoadRunner: \texttt{rrrun test.Test}.

To launch the contract analyzer on a~simple program, run the following commands:
\begin{lstlisting}
$ javac test/ContractTest.java
$ rrrun -tool=CT -contractFile=test/ContractTest.contract test.ContractTest
\end{lstlisting}
The analyzer should find a~contract violation.

\chapter{Contract Definition Grammar}
\label{appGrammar}

The configuration files with contract definitions must follow the grammar
presented below. The grammar is described in the BNF syntax.

\begin{lstlisting}
<contract> ::= <clause> | <contract> <clause>

<clause> ::= <method_expr> "<-" <method_expr> ";"

<method_expr> ::= <method>
                | "(" <method_expr> ")"
                | <method_expr> "|" <method_expr>
                | <method_expr> "(" <method_expr> ")"
                | <method_expr> <method>

<method> ::= <class> <name> <descriptor> <metavars>

<metavars> ::= <metavar> "=" <metavar> "(" <metavars_list> ")"
    | <metavar> "=" <metavar> "()" 
    | <metavar> "=("  <metavars_list> ")"
    | <metavar> "(" <metavars_list> ")"
    | "(" <metavars_list> ")"
    | <metavar> "()" 
    | "()"

<metavars_list> ::= <metavar> | <metavars_list> "," <metavar>

<metavar> ::= <letter> | "_"
\end{lstlisting}

The terminals have the following definitions:
\begin{itemize}
    \item \texttt{<class>} is a~class name in the internal representation of
        the JVM specification, for example: \texttt{java/lang/Object};
    \item \texttt{<name>} is the method name, as defined by the JVM
        specification, for example: \texttt{equals};
    \item \texttt{<descriptor>} is the method descriptor, as defined by JVM
        specification, for example: \texttt{(Ljava/lang/Object;)Z};
    \item \texttt{<letter>} is a~single letter, such as \texttt{X}.
\end{itemize}

\chapter{Class Diagram of the Contract Analyzer}
\label{diagram}

The following page contains a~UML diagram of the contract analyzer described in
Chapter~\ref{chThree}. In the diagram, the following classes are omitted for
clarity:
\begin{itemize}
    \item the \texttt{ContractTool} class which connects
    \texttt{ContractAnalyzer} to RoadRunner,
    \item the contract lexer and parser,
    \item basic classes and interfaces used for holding simple data, such as
    \texttt{Args}, \texttt{Signature}, \texttt{MetaVars},
    \texttt{ImmutableVectorClock}, and \texttt{ContractParams}.
\end{itemize}

The \texttt{ContractAnalyzer} class is a~generic class with two type parameters:
\texttt{T} is a~type representing a~thread, \texttt{L} represents a~lock. During
the analysis, RoadRunner types are used: \texttt{T} is \texttt{ShadowThread},
\texttt{L} is \texttt{ShadowLock}. Collections and containers used in the
diagram are parts of the Vavr library.

\pagebreak

\begin{tikzpicture}[every node/.style={scale=0.8}]
% \draw[help lines,step=5mm,gray!20] (-1,-16) grid (14,1);
\umlclass[anchor=north west, template={T,L}]{ContractAnalyzer}{
    -- Contract \\
    -- Decoration<T, Window> \\
    -- Decoration<T, ImmutableVectorClock> \\
    -- Decoration<L, ImmutableVectorClock> \\
    -- getTid: Function<T, Integer> \\
    -- threads: List<T>
}{
    + exit(T, Signature, Args) \\
    + create(T) \\
    + acquire(T, L) \\
    + release(T, L) \\
    + fork(T, T) \\
    + join(T, T)
}
\umlclass[x=0, y=-6, anchor=north west, type=interface]
{InstanceCollection}{
}{
    + getByState(State): Traversable<Instance> \\
    + advance(Function<Instance, Option<\\
    \quad \quad Either<ViolationException, Instance>{}>{}>): \\
    \quad \quad Either<ViolationException, InstanceCollection>
}
\umlclass[x=15.25, y=-4, anchor=north east]{Window}{
    -- tid: int \\
    -- targets: InstanceCollection \\
    -- spoilers: InstanceCollection
}{
    + Window(tid:int, Contract) \\
    + exit(Signature, Args, Seq<Window>, \\
    \quad \quad ImmutableVectorClock): \\
    \quad \quad Either<ViolationException, Window>
}
\umlclass[x=15.25,y=0,anchor=north east]{Contract}{
    -- targets: Map<State, State> \\
    -- signatures: Set<Signature>
}{
    + Contract(Map<State, State>) \\
    + contains(Signature): boolean \\
    + getTargets(): Map<State, State> \\
    + merge(Contract): Try<Contract>
}
\umlclass[x=0.75, y=-9.5, anchor=north west]
{MultimapInstanceCollection}{
    -- instances: Multimap<State, Instance>
}{
    + MultimapInstanceCollection( \\
    \quad \quad states: Map<State, State>)
}
\umlclass[x=15.25, y=-8, anchor=north east]{Instance}{
    -- fa: FiniteAutomaton \\
    -- conflicting: State \\
    -- lastBegin: Option<ImmutableVectorClock> \\
    -- lastEnd: Option<ImmutableVectorClock> \\
    -- begin: Option<ImmutableVectorClock>
}{
    + Instance(FiniteAutomaton, State) \\
    + advance(Signature, Args, ImmutableVectorClock, \\
    \quad \quad onAccepted: Function2<Instance, State, \\
    \quad \quad Option<ViolationException>{}>): \\
    \quad \quad Option<Either<ViolationException, Instance>{}> \\
    + isViolatedBySpoiler(spoiler: Instance, tid: int, \\
    \quad \quad targetTid: int, vc: ImmutableVectorClock) \\
    + violatesTarget(target: Instance, tid: int, \\
    \quad \quad targetTid: int, vc: ImmutableVectorClock) \\
    + matches(Instance): boolean \\
    + isRunning(): boolean \\
    + hasParams(): boolean
}
\umlclass[x=0, y=-14, anchor=north west]{FiniteAutomaton}{
    -- start: State \\
    -- current: State \\
    -- params: Option<ContractParams>
}{
    + FiniteAutomaton(State) \\
    + advance(Signature, Args): \\
    \quad \quad Option<FiniteAutomaton> \\
    + reset(): FiniteAutomaton \\
    + paramsMatch(FiniteAutomaton): boolean \\
    + isAccepted(): boolean \\
    + isRunning(): boolean \\
    + hasParams(): boolean
}
\umlclass[x=15.25, y=-16, anchor=north east]{State}{
    -- transitions: Map<Signature, \\
    \quad \quad Tuple2<State, MetaVars>{}>
}{
    \umlstatic{+ last(): State} \\
    \umlstatic{+ of(transitions: Map<Signature, Tuple2<State,} \\
    \quad \quad \umlstatic{MetaVars>{}>): Try<State>} \\
    + getNextState(Signature, Args): Option<Tuple2<\\
    \quad \quad State, ContractParams>{}> \\
    + getSignatures(): Set<Signature> \\
    + concatenate(State): State \\
    + alternate(State): Try<State>
}
\umlaggreg[mult2=1, anchor1=28, anchor2=172]{ContractAnalyzer}{Contract}
\umlaggreg[mult2=*, anchor1=-29, anchor2=155]
{ContractAnalyzer}{Window}
\umlcompo[arg2=targets, mult2=1, anchor1=-168, anchor2=17]
{Window}{InstanceCollection}
\umlcompo[arg2=spoilers, mult2=1, anchor1=-155, anchor2=5]
{Window}{InstanceCollection}
\umlaggreg[mult2=*, anchor1=-17, anchor2=141]{InstanceCollection}{Instance}
\umlimpl{MultimapInstanceCollection}{InstanceCollection}
\umlcompo[mult2=1, anchor1=-140, anchor2=33]{Instance}{FiniteAutomaton}
\umluniassoc[arg2=conflicting, mult2=1]{Instance}{State}
\umluniassoc[arg2=start, mult2=1, anchor1=0, anchor2=151]
{FiniteAutomaton}{State}
\umluniassoc[arg2=current, mult2=1, anchor1=-15, anchor2=163]
{FiniteAutomaton}{State}
\umluniassoc[mult2=*, loopsize=1.5cm, angle1=-155, angle2=-140]{State}{State}
\end{tikzpicture}
