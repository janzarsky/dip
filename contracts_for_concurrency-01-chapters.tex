\chapter{Introduction}

\todo{}

\chapter{Testing and Dynamic Analysis in Java}

\todo{}

\chapter{Contracts for Concurrency}

\todo{chapter contents}

When developing software, one frequently uses modules created by someone else
via it's programming interface. For example, in object oriented programming, the
interface consists of public methods of given class. Accessing the interface
requires one to follow a protocol consisting of: (i) syntax, i.e. types of
parameters and return values, (ii) semantics, i.e. the expected behavior for
given input parameters, and (iii) access restrictions. Access restrictions
include the domain of valid values, dependencies on other services, and
atomicity violations \cite{FITPUB11510}.

\emph{Contracts for concurrency} \cite{FITPUB10817}
\cite{DBLP:journals/corr/SousaDFL15} are a case of a software protocol that
expresses access restrictions in concurrent setting. In it's basic form, it
specifies sequences of methods that must be executed atomically. The contracts
can be extended with parameters to reflect the data flow between the methods (so
that only methods manipulating the same data must be executed atomically).
Another extension adds so called \emph{spoilers} (so that given sequence must be
executed atomically only with respect to only certain sequences). Both
extensions can be combined \cite{FITPUB11510}.

\todo{
  * testing of parallel applications, dynamic vs. static analysis
  * logical clock
  * happens-before
  * contracts for concurrency
  * parametric contracts
  * spoilers
  * dynamic analysis description
  * optimizations
  * trace windows, discarding spoilers and targets
  * vector clocks
  * roadrunner framework
  * instrumentation
  * ASM library
}

\chapter{Dynamic Analyzer for Parametric Contracts}

\todo{}

\chapter{Conclusion}

\todo{}
